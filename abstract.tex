%
%  Abstract
%

\begin{abstract}

\addcontentsline{toc}{chapter}{Abstract}

%todo: max 350 words
Web applications are an integral part of our lives and culture. We use
web applications to manage our bank accounts, interact with friends,
and file our taxes. A single vulnerability in one of these web
applications could allow a malicious hacker to steal your money, to
impersonate you on Facebook, or to access sensitive information, such
as tax returns. It is vital that we develop new approaches to discover
and fix these vulnerabilities before the cybercriminals exploit them.

In this dissertation, I will present my research on securing the web
against current threats and future threats. First, I will discuss my
work on improving black-box vulnerability scanners, which are tools
that can automatically discover vulnerabilities in web applications.
Then, I will describe a new type of web application vulnerability:
Execution After Redirect, or EAR, and an approach to automatically
detect EARs in web applications. Finally, I will present deDacota, a
first step in the direction of making web applications secure by
construction.

% Wil did not have an abstract signature
% \abstractsignature

\end{abstract}




%\begin{germanabstract}

%\addcontentsline{toc}{chapter}{Zusammenfassung}

%\end{germanabstract}


