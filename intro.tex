\def\currentprefix{intro}

\lstset{ %
language=Php,                % the language of the code
basicstyle=\ttfamily,       % the size of the fonts that are used for the code
escapeinside={\%*}{*)},         % if you want to add a comment within your code
morekeywords={*,...},           % if you want to add more keywords to
}


%
%  Introduction
%

%-- Chapter Title
\chapter{Introduction}

Web applications are a fundamental part of our lives and culture. We
use web applications in almost every facet of society: socializing,
banking, health care, taxes, education, and news, to name a few. These
web applications are always available from anywhere with an Internet
connection, and they enable us to communicate and collaborate at a
speed that was unthinkable just a few decades ago.

The shift of having more and more of our lives and data move to web
applications unfortunately attracts hackers. In 2011, hackers stole 1
million usernames and passwords from Sony~\cite{bilton11:sony}. In
2007, hackers stole 45 million customer credit cards from TJ
Maxx~\cite{jewell07:tjmaxx}. In 2009, hackers stole 100 million
customer credit cards from Heartland Payment
Systems~\cite{acohido09:heartland}. In 2012, hackers stole 24,000
Bitcoins\footnote{These Bitcoins are worth about \$10 million at this
  time of writing.} from BitFloor, a major Bitcoin
exchange~\cite{kirk12:bitfloor}. What all of these instances have in
common is that hackers exploited vulnerabilities in a web application
to steal either usernames and passwords, credit cards, or Bitcoins.

It is insightful to turn our attention to \emph{why} web applications
have so many vulnerabilities. Those same properties that make web
applications so attractive to users also attract hackers. A web
application never closes, so they are always available for hackers.
Web applications also house a vast treasure-trove of data, which
hackers can use for monetary gain. Finally, as we will explore in the
next section, web applications are a complex hodgepodge of various
technologies. This complexity, combined with the intense
time-to-market pressure of web applications, is a breeding ground for
bugs and vulnerabilities.

The situation is dire. We must focus on new ways to secure web
applications from attack. We must develop new tools in order to find
the vulnerabilities before a hacker does. We must, because web
applications and the data they store are too important.

\section{History of Web Applications}

The World Wide Web was created by Sir.\ Tim Berners-Lee in 1989 as a
means of sharing information for the CERN research organization. What
began as a means of sharing simple hyper-linked textual documents
quickly exploded in popularity over the proceeding years.

The core of the web has remained relatively the same throughout the
years: a web browser (operated by a user) connects to a web server
using the Hypertext Transfer Protocol (HTTP)~\cite{fielding99:http11}.
The web server then sends back a response, typically in the form of a
HyperText Markup Language (HTML) page~\cite{berjon14:html5}. The web
browser then parses the raw HTML page create a graphical web page that
is displayed to the user. The fundamental underlying principle, and
the definition of HyperText, is that an HTML page contains links to
\emph{other} HTML pages.

The beginning of the web was envisioned as a set of documents with
links pointing to other documents. In other words, the web was mostly
a set of read-only documents (from the perspective of the user with
the web browser). This is where the term \emph{web site} comes from: a
web site is typically thought of as a collection of documents that
exist under the same domain name.

As the web evolved, web sites started to shift from static, read-only
documents. Developers realized that the HTML response returned to the
client could be dynamic---that is, the content of the HTML response
could vary programmatically. This shift in thinking caused web sites to
transition to \emph{web applications} which emulated features of
traditional desktop applications. Web applications enabled scenarios
that caused the web's popular to increase: e-commerce, news sites, and
web-based email. It is hard to overstate the impact that web
applications had on the uptake of the web. 

Now, with web applications, the architecture of the web changed. When
the web browser makes an HTTP request to the server, instead of
returning a static HTML response, the web server typically will invoke
server-side code. This server-side code is responsible for returning a
response, typically HTML, to the browser. The server-side code can use
any number of inputs to determine the response, but the typical set-up
is that the server-side code reads the parameters sent in the
browser's HTTP request, consults an SQL database, and returns an HTML
response.

This architecture of a web application is what we will use throughout
this chapter to discuss the security aspects of web applications.

\section{Web Application Vulnerabilities}

The security properties of a web application are similar to the
security of any other software system: confidentially of the data,
integrity of the data, and availability of the application. In this
dissertation, we will focus on attacks that compromise the
confidentially or integrity of the web application's data.

\subsection{Injection Vulnerabilities}

Injection vulnerabilities are to web applications as memory-corruption
vulnerabilities are to binary applications. 

This class of vulnerabilities occur when an attacker is able to
control or influence the value of parameters that are used as part of
an outside (to the language) query, command, or language. If the
attacker can manipulate and change the semantics of the query,
command, or language, and this compromises the security of the
application, then that is an injection vulnerability.

There are many types of injection vulnerabilities in web applications,
and the types depend on the query, command, or language that is being
injected. These include SQL queries, HTML responses, OS commands,
email headers, HTTP headers, and many more. Next we will focus on two
of the most serious and prevalent classes of injection
vulnerabilities: SQL injection and Cross-Site Scripting (XSS).

\subsubsection{SQL Injection}

SQL injection vulnerabilities, while declining in the number reported
compared to XSS vulnerabilities, are still numerous and are incredibly
critical when they occur.

The root cause of SQL injection vulnerabilities is that the
server-side code of the web application, in order to issue an SQL
query to the SQL database, concatenates strings together. This format
allows the queries to be parameterized, and therefore the server-side
code can be more general. 

{\ssp
\begin{lstlisting}[language=php, caption={[Example of an SQL injection
        vulnerability.]Example of an SQL injection vulnerability in a
      PHP web application. The attacker-controlled \texttt{\$name} parameter is used
      unsanitized in the SQL query created on Line~2 and issued on Line~3.},
    label=\currentprefix:sql-injection, float,]
$name = $_GET['name'];
$q = "select * from users where name = '" . $name . "';";
$result = mysql_query($q);
\end{lstlisting}
}


The code in Listing~\localref{sql-injection} shows a sample PHP web
application that contains an SQL injection vulnerability. In Line~1 of
this sample, the variable \texttt{\$name} is set based on the value of
the query parameter called \texttt{name}. The \texttt{\$name} variable
is used in Line~2 to construct an SQL query to look up the given user
by name in an SQL table. The web application issues the query on Line~3.

The problem is that, according to the server-side language, the
resulting query is simply a string, whereas when that string is passed
to the SQL server, the SQL server parses the string into a SQL query.
Therefore, what the server-side code treats as a simple string is in
fact a complex language. 

In Listing~\localref{sql-injection}, the vulnerability comes from the
fact that the query parameter \texttt{name} comes from the user and is
untrusted. As seen in the example, the \texttt{\$name} variable is
used in the SQL query to select based on the SQL \texttt{name} column.
In order to do this, the programmer constrains the value to be
in between matching \texttt{\textquotesingle}. Therefore, for the
attacker to alter the semantics of the query, the attacker need only
input something like the following: \texttt{\textquotesingle or 1=1;
  \#}. This would cause the resulting query that the web application
issues to the database to be the following:

\texttt{select * from users where name =
  \textquotesingle\textquotesingle or 1=1; \#\textquotesingle;}

The \texttt{\#} is an SQL comment which means that everything after
that in the query is ignored. Now the attacker has been able to alter
the semantics of the SQL query, in this case by adding another SQL
clause that was not in the original statement. 

Thus, in order for an attacker to not alter the semantics of the SQL
query, a web developer must be careful to properly \emph{sanitize} all
potentially attacker-control input. Here sanitize means to transform
the input from the user to a form that renders it neutral for the
target language. In the case of SQL, this typically means converting
any \texttt{\textquotesingle} (which are used by an attacker to escape
out of an SQL query) to the inert
\texttt{\textbackslash\textquotesingle}.

So what damage can an attacker cause by exploiting an SQL injection
vulnerability? An attacker can violate both the confidentially and
integrity of the application's data. An attacker can insert arbitrary
data into the database, potentially adding a new admin user to the web
application. Also, an attacker can exfiltrate any data that the
database user can access (typically all data that the application can
access). Finally, the attacker can also delete all of the data that
the web application has access to. These consequences of a single SQL
injection vulnerability are why SQL injection vulnerabilities are so
severe.

To prevent SQL injections with sanitization, a developer must be
extremely careful that no user-supplied data is used in an SQL
statement, including any paths that the data could have taken through
the web application. In practice, this is (understandably) difficult
for developers to always accomplish. Therefore, developers should use
\emph{prepared statements,} which is a way to tell the database the
structure of an SQL query \emph{before} the data is given. In this
way, the database already knows the structure of the SQL query, and
therefore there is no way for an attacker to alter the structure and
semantics of the query. Almost every server-side language or framework
has support for prepared statements.

\subsubsection{Cross-Site Scripting}

Cross-Site Scripting (XSS) vulnerabilities are similar in spirit to
SQL injection vulnerabilities. Instead of injection into a SQL
queries, XSS vulnerabilities are injections into the HTML output that
the web application generates. XSS vulnerabilities are frequently in
the top three of all reported vulnerabilities in \emph{all} software
systems.

The root cause of XSS vulnerabilities is that the server-side code of
a web application, in order to create the web application's HTML
response, essentially concatenates strings together.

{\ssp
\begin{lstlisting}[language=php, caption={[Example of a XSS 
        vulnerability.]Example of a XSS vulnerability in a
      PHP web application. The attacker-controlled \texttt{\$name} parameter is used
      unsanitized in the HTML output on Line~2.},
    label=\currentprefix:xss-example, float,]
$name = $_GET['name'];
echo "Hello <b>" . $name . "</b>";
\end{lstlisting}
}


Listing~\localref{xss-example} shows an example PHP web application
that has an XSS vulnerability. In Line~1, the variable \texttt{\$name}
is retrieved from the query parameter \texttt{name}. Then,
\texttt{\$name} is used in Line~2 as an argument to PHP's
\texttt{echo} function, which sends its string argument to the HTTP
response. The goal of this code is to output the user's name in bold.
This is accomplished in HTML by wrapping the user's name in a bold tag
(\texttt{<b>}).

If an attacker is able to control the HTML output of the web
application, as the \texttt{\$name} parameter in
Listing~\localref{xss-example}, then the attacker can trick the user's
web browser into executing the attacker's JavaScript. This can be
accomplished in a variety of ways, one example would be inputting the
following for the \texttt{name} query parameter:

\texttt{<script>alert('xss');</script>}

This input is the way for the web application to tell the user's
browser to execute JavaScript. 

The fundamental building block of JavaScript security in the web
browser is the \emph{Same Origin Policy}. In essence, this security
policy means that only JavaScript that comes from the same
origin\footnote{Here, we leave the exact description of same origin to
  be imprecise. We will define it later when necessary.} can interact.
In practice, what this means is that JavaScript running on the browser
from \texttt{hacker.com} cannot interact with or affect JavaScript
running on the browser from \texttt{example.com}.

The name \emph{Cross-Site Scripting} is derived from the fact that XSS
circumvents the browser's Same Origin Policy. By using an XSS
vulnerability, an attacker is able to trick a user's browser to
execute JavaScript code of their chosen in the web application's
origin. This is because, from the browser's perspective, the
JavaScript came from the web application, so the browser happily
executes it along with the web application's JavaScript.

Armed with an XSS vulnerability, a hacker can wreck havoc to
compromise a web application. A popular XSS exploitation technique is
to steal the web application's cookies and send them to the attacker.
Typically the web application's cookies are used to authenticate and
keep state with the web application, which might allow the attacker to
impersonate the user. 

By executing JavaScript in the same origin as the web application, the
attacker's JavaScript has total control over the DOM of the web page.
What this means is that the attacker can completely alter the look of
the web page, and could, for instance, change the page into the web
application's login form. However, once the user put their information
into the form, the attacker's JavaScript could steal that information.
In this way, the attacker is able to phish the user's credentials,
except in this instance the user is actually on the correct domain
name.

The other insidious thing that an attacker's JavaScript can do if it
executes in the user's browser is interact with the web application on
behalf of the user. This defeats any CSRF protection that the web
application has enabled, as the attacker's JavaScript can read all the
CSRF tokens. In practice, what this means is that the attacker's
JavaScript can interact with the web application, and the web
application has no way of knowing that the requests did not come from
the user. Imagine an attacker's JavaScript sending emails as you or
initiating a bank transfer.

XSS vulnerabilities can be fixed by proper sanitizaiton at all program
points in the web application that output HTML. This sanitization
process typically will convert entities that are significant in
parsing HTML to their display equivalent. For instance, the HTML
\texttt{<} character is transfer to its HTML entities equivalent
\texttt{\&gt;}, which means to display a \texttt{<} character on the
resulting web page, rather than starting an HTML tag.

There are a number of practical difficulties that make properly
sanitizing output for XSS vulnerabilities particularly challenging
(especially when compared to SQL injection vulnerabilities). One is
that, as shown by Saxena, Molnar, and
Livshits~\cite{saxena11:scriptgard}, there are numerous types of
sanitization for XSS vulnerabilities, and which type of sanitization
to use depends on \emph{where} the output is used in the resulting
HTML page. This means that the developer must reason not only about
all program paths that a variable may take to get to a specific point
(to see if an attacker can influence its value), but also about all
the different places in the HTML output where the variable is used.
With all this complexity, it is clear why XSS vulnerabilities are
still the most frequent web vulnerability. 

Unfortunately XSS vulnerabilities have no easy, widely supported fix,
as prepared statements are to SQL injection vulnerabilities. However,
in Chapter~\ref{dedacota} we will look at an approach to fundamentally
solve XSS vulnerabilities.

\subsection{Logic Flaws}

As the name suggests, logic flaws are a vulnerability class that
occurs when the logic of the web application does not match with the
developer's intended logic of the web application. One popular example
would be, on an ecommerce application, if a user is able to submit a
coupon multiple times, until the price of the item is zero. 

An injection vulnerability can affect any web application, and the
remedy of the vulnerabilities will be the same, regardless of the
underlying web application. In contrast, logic flaws are specific and
unique to the web application. Behavior that appears in two web
applications may be a logic flaw in one but a security vulnerability
in the other. The distinguish feature of logic flaw vulnerabilities is
that the web application code is functioning correctly, however that
behavior violates the developer's security model of the application.
Therefore, these vulnerabilities are incredibly difficult to detect in
an automated fashion, as the automated tool must reverse engineer the
developer's intended security model.

In Chapter~\ref{fear-the-ear}, we will describe a novel class of logic
flaw vulnerabilities called Execution After Redirect.

\section{Securing Web Applications}

Given the rise in popularity, ensuring that web applications are
secure is critical. Security flaws in a web application could allow an
attacker unprecedented access to secret data. 

One approach is to either prevent vulnerabilities in web applications
in the first place, or to detect when an attack takes place. 

\subsection{Anomaly Detection}

One way to secure web applications is to have tools and approaches
that look for attack traffic on against the web application. These
approaches typically 

\subsection{Secure Construction}

\section{Vulnerability Analysis Tools}

Vulnerability analysis is the art of actively finding vulnerabilities
in software. The idea here is to find vulnerabilities either before an
application is deployed or before a hacker is able to find the
vulnerability. 

Manual vulnerability analysis is when a team of humans manually
analyze an application for vulnerabilities. These vulnerability
analyses, frequently called \emph{pentesting,} are great at finding
vulnerabilities in software. However, the downside is that a expert's
time is costly, therefore, due to the cost, a company very
infrequently does pentesting of its applications.

Vulnerability analysis tools are automated approaches to find
vulnerabilities in software. The goal of this type of software is to
find all possible vulnerabilities in an application. The idea is to
encapsulate a human security expert's knowledge into a program that
can be reused. 

Because vulnerability analysis tools are automated, they can be used
against a variety of applications. Furthermore, they are much less
expensive than hiring a human expert, so they can be used much more
frequently throughout the software development process.

Vulnerability analysis tools can be categorized based on what
information of the web application they use. In the following
sections we will describe the difference between a white-box,
black-box, and grey-box vulnerability analysis tools. 

\subsection{White-Box}

A white-box vulnerability analysis tool looks at the source code of
the web application to find vulnerabilities. By analyzing the source
code of the web application, a white-box tool can see \emph{all}
program paths throughout the application. This enables a white-box
tool to potentially find vulnerabilities along all program paths.
Typically approaches leverage ideas and techniques from the program
analysis and static analysis communities to find vulnerabilities.

\textbf{How do they actually find vulns?}

The biggest strength of white-box tools is that they are able to see
all possible program paths through the application. However, as
precisely finding all vulnerabilities in an application via static
analysis is equivalent to the halting problem, trade-offs must be made
in order to create functional tools. The trade-off that is often made
in white-box tools is one of being complete rather than precise. What
this means is that a white-box tool will report vulnerabilities that
are not actual vulnerabilities. This is usually because static
analysis will over-approximate the program paths that the application
can take. Thus, there will be vulnerabilities reported that cannot
occur in practice. 

The downside of white-box tools is that they are tied to the specific
language or framework. A white-box vulnerability analysis tool written
for PHP will not work for Ruby on Rails without significant
engineering work. These tools are tightly coupled to not only language
features, but also framework features. 

\subsection{Black-Box}

In contrast to white-box tools, black-box vulnerability analysis tools
assume no knowledge of the source-code of the web application. Instead
of using the source code, black-box tools interact with the web
application being tested just as a user with a browser would.
Specifically, this means that the black-box tools issue HTTP requests
to the web application, and receive back HTTP responses containing
HTML. These HTML pages tell the black-box tool how to generate new
HTTP requests to the application. 

Black-box tools first will crawl the web application looking for all
possible \emph{injection vectors} into the web application. An
injection vector is any way that an attacker can get input into the
web application. In practice, web application injection vectors are
things like URL parameters, HTML form parameters, HTTP cookies, HTTP
headers, URL path, and so on. 

Once the black-box tool has enumerated all possible injection vectors
in the application, the next step is to give the web application input
which is intended to trigger or expose a vulnerability in the web
application. This process is typically called \emph{fuzzing.} The
specifics of choosing which injection vectors to fuzz and when are
specific to each black-box tool. 

Finally, the black-box tool will judge the HTML and HTTP response to
the fuzzing attempts in order to tell if they were successful. If they
are, the black-box tool will report it as a vulnerability. 

There are two major benefits of black-box tools as opposed to
white-box tools. The first is that black-box tools are general and can
find vulnerabilities in \emph{any} web application, regardless of what
language the server-side code is written in. In this way, black-box
tools emulate an external hacker who has no access to the source code
of the application. Therefore, black-box tools are applicable to a
much larger number of web applications. 

The second major benefit is that black-box tools have significantly
lower false positives than white-box tools. This is because the
fuzzing attempt actually tries to trigger the vulnerability, and, for
most web vulnerabilities, a successful exploitation will be evident in
the resulting HTML page. Ultimately, what this means is that
developers who run these tools against their own web applications
trust the output of a black-box tool than a white-box tool. 

The drawback of a black-box tool is that it is not guaranteed to find
all vulnerabilities in your web application. This is because a
black-box tool can only find vulnerabilities along program paths that
it executes, whereas a white-box tool can see all program paths
through an application. 

\subsection{Grey-Box}

As the name suggests, grey-box tools are a combination of white-box
techniques and black-box techniques. The main idea is to use white-box
static analysis techniques to generate possible vulnerabilities. Then,
there is a confirmation step where the tool will actually try to
exploit the vulnerability. Only if this step is successful will the
tool report the vulnerability.

Grey-box tools inherit the pros of white-box tools: ability to find
vulnerabilities in all program paths along with the low false positive
rate associate with black-box tools (as the vulnerabilities are
verified by black-box techniques). However, grey-box tools also
inherit the cons of white-box tools: applicability to a single web
application language or framework. Therefore, these types of tools are
not very popular.



% --------------------------------------------------------------------
% Thesis Statement
% --------------------------------------------------------------------
%% \pagebreak
%% \section*{Thesis Statement}

%% Is this something I need? Did wkr have it?



